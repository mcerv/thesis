\chapter*{Conclusion}
Diamond is a viable choice as a particle detector due to its durability and versatility. Its unique properties allow it to be used in a number of applications ranging from high-energy physics experiments and beam monitors to neutron spectrometers. 
%This thesis shows two of the applications and the results of an irradiation study to showcase its operational stability after high irradiation.

An irradiation study has been carried out to define the measurement limitations for sCVD diamond in terms of temperature and irradiation damage. A damage factor $k_{\mathrm{\lambda}}=(4.4\pm1.2)\times10^{-18}~\upmu$m$^{-1}$~cm$^{2}$ has been obtained for 300~MeV $\uppi$ irradiated single-crystal CVD diamonds, a factor two higher than the comparable RD42 irradiation study. A long-term stability for $\upbeta$ and $\upalpha$ particle measurements with irradiated diamonds is presented, showing a stable charge collection of the former after a few million $\upbeta$ counts and a significant decrease in charge collection for the latter after a few thousand $\upalpha$ counts. To counteract the deteriorating $\upalpha$ measurements, several ``healing'' procedures are proposed to recover the charge collection to the initial value. TCT measurements of the long-term $\upalpha$ measurements hint at a development of complex internal electric fields due to accumulated space charge in the bulk of the sensor.

$\upalpha$-TCT measurements of the sCVD diamond sensors before and after irradiation show that charge trapping time increases linearly with irradiation. The acquired values for holes and electrons $\beta_\mathrm{e}=(3.8\pm0.9)\times10^{-16}~$cm$^2$/ns and $\beta_\mathrm{h}=(3.4\pm0.8)\times10^{-16}~$cm$^2$/ns are a factor two lower than those for silicon, which means that the same fluence creates twice as many charge traps in silicon than it does in diamond.

$\upalpha$-TCT measurements of non-irradiated sCVD diamond between 4~K and 295~K (room temperature) conducted in liquid helium show that the non-irradiated diamond collects only a third of the charge at 4~K as compared to room temperature. The results are consistent with those reported by H. Jansen. Additional measurements of irradiated diamond show that the charge collection efficiency decreases with irradiation, but the relative decrease in efficiency is lower below 75~K. This is probably due to higher drift velocity of electrons and holes at low temperatures, which reduces the probability of charge trapping.

Two applications for particle detection with CVD diamond sensors are presented, the first by monitoring the collected charge and the second by monitoring the current.

The ATLAS Diamond Beam Monitor is the first pixelated diamond detector used in high-energy physics experiments. Its goal is to measure bunch-by-bunch luminosity in the ATLAS experiment at CERN. It has been designed as an upgrade of the existing diamond-based luminosity detector, the Beam Condition Monitor. It is instrumented with 24 polycrystalline CVD sensors grouped in 8 telescopes that are positioned around the interaction point in the ATLAS experiment. This geometry that points toward the interaction point, together with the high spatial segmentation of the pixelated sensors, allows it to distinguish between the beam background and the tracks of particles produced by proton collisions. The thesis describes the assembly and testing procedure of the detector, as well as integration and commissioning stages. Finally it shows the first data obtained with proton-proton collisions in the LHC.

The second application uses sCVD diamond for selective spectroscopic measurements. It uses CIVIDEC detectors and readout to carry out real-time particle identification. The system analyses ionisation profiles of the current pulses induced in diamond by incident particles and determines the type of radiation by the shape of the profiles. It is capable of analysing particles at a rate up to $6\times10^6$~s$^{-1}$, which is a significant upgrade from the existing offline data analysis since it reduces the dead time of the measurement. By use of user-defined qualifiers it determines the radiation type and shows the energy spectrum of the accepted and rejected radiation type. The application is tailored for measurements in high-flux environments with mixed types of radiation, such as neutron reactors. The thesis presents several use cases with data collected in CROCUS at EPFL, Switzerland and TRIGA at Atominstitut, Austria. 
