% ---------------------------- comment out when compiling the full document -------------------
\documentclass[12pt]{mytustyle}  % default square logo 
\usepackage{amssymb}
\usepackage{caption}
\usepackage{upgreek}
\usepackage{subfig}

\usepackage{packages/fancyhdr}% http://ctan.org/pkg/fancyhdr
\pagestyle{fancy}% Change page style to fancy
%\fancyhf{}% Clear header/footer
\fancyhead{}
\fancyhead[RO, LE]{Current monitoring}
%\fancyfoot{}
%\fancyfoot[RO, LE]{\thepage}% \fancyfoot[R]{\thepage}
\renewcommand{\headrulewidth}{0.7pt}% Default \headrulewidth is 0.4pt
%\renewcommand{\footrulewidth}{0.7pt}% Default \footrulewidth is 0pt
%\rfoot{\thepage}

\begin{document}
\baselineskip=16pt
% ---------------------------------------------------------------------------------------------------------------



% ---------------------------------------------------------------------------------------------------------------
\chapter{Current monitoring}
% ---------------------------------------------------------------------------------------------------------------
	- Real-time particle identification
		- Pulse shape (width, area. . .) and its constrants
		- Device constraints
		- Vienna TRIGA reactor
//		- Real time cross section measurement at IRMM?

% ---------------------------------------------------------------------------------------------------------------
\clearpage
\section{Neutron research institutes and collaborations}
\label{sec:resinstcol}
% ---------------------------------------------------------------------------------------------------------------

\subsection{Atominstitut, Vienna}
Atominstitut (ATI), an institute for atomic and subatomic physics, was established in 1958 in Vienna as an inter-university institute. It currently houses around 200 people involved in a broad range of research fields: quantum, particle, neutron, nuclear, radiation and reactor physics, quantum optics etc. Its central facility is a TRIGA MARK II neutron reactor (described in detail below). 

As of 2002 the ATI is part of the University of Technology in Vienna.


\subsubsection{TRIGA MARK II neutron reactor}
TRIGA MARK II is a reactor of a swimming-pool type used for training, research and isotope production. It is one of 40 such reactors worldwide, produced by an californian company General Atomic in the early 60's. It is capable of continuous operation at a maximum output power of 250~kW. 

The reactor core consists of 3~kg of 20~\% enriched uranium ($^{235}$U). The fuel moderator rods are mostly made up of zirconium with low percentage of hydrogen and uranium. Both the core and the rods are immersed in a pool of water for the purpose of cooling and radiation protection. The surrounding concrete walls are 2~m wide with an added graphite layer for improved shielding. Four main experimental beam holes are placed radially through the walls. All exits are heavily shielded to prevent radiation damage to people, but still leaving enough space to set up experiments. Apart from the beam holes, there are several other exits and components, e.g. a thermal column for generation of thermal (low energetic) neutrons.

% MORE TOMORROW!





\subsection{n-ToF, CERN}
n-ToF (or neutron time-of-flight) is a scientific collaboration with the aim of studying neutron-nucleus interactions. Over 30 institutes and universities are currently active members of this collaboration, among them Atominstitut in Vienna.

n-ToF is also a facility at CERN where the experiments are carried out in a 200 m long experimental area. The knowledge stemming from the experimental results can then be applied in various fields ranging from nuclear technology and cancer therapy to astrophysics.

A pulsed beam of highly energetic protons (20~GeV/c) is produced by the Proton Synchrotron (PS) and aimed at a fixed lead spallation target. Each proton hitting the target produces around 300 neutrons of various energies. Initially highly energetic neutrons are slowed down by the target and by a slab of water placed behind it. This broadens their energy spectrum, which then ranges from meV (so-called thermal neutrons) to GeV (so-called fast neutrons). The neutrons are then collimated and sent through a 185~m long evacuated pipe to the experimental area, where they are made to collide with another target or a sample. The radiation resulting from the collisions is detected by a set of dedicated detectors around the interaction point. Having different energies, neutrons travel with different speeds, highly energetic ones reaching the target faster than those with low energies. Analysis of the collisions with a precise timing allows us to determine the interaction probability with sample material as a function of incident neutron energy.






% ---------------------------------------------------------------------------------------------------------------
\clearpage
\section{Real-time particle identification}
\label{sec:rtpi}
% ---------------------------------------------------------------------------------------------------------------


\subsection{Pulse parameters}
\subsection{Real-time pulse shape analysis algorithm}
\subsection{Device specifications and constraints}
Lab measurements
\subsection{Performance results}
\subsubsection{Thermal neutron measurements}


%\chapter{Particle discrimination}
%
%\section{Discrimination based on TCT pulses}
%
%\subsection{Ionisation profiles in diamond}
%
%
%\subsection{Discrimination parameters}
%FWHM, TCT falling edge (2 pg)
%
%
%\section{Real-time pulse parametrisation}
%(4 pg)
%
%\subsection{Readout device}
%(2 pg)
%
%\subsection{Pulse shape analysis algorithm}
%(8 pg)
%
%\subsection{Device specifications}
%(5 pg)
%
%\subsection{Application: Thermal neutron counter}
%(10 pg)
%
%
%\section{}




% ---------------------------- comment out when compiling the full document -------------------
\end{document}
% ---------------------------------------------------------------------------------------------------------------
