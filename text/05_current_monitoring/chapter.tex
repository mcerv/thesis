% ---------------------------- comment out when compiling the full document -------------------
\documentclass[12pt]{mytustyle}  % default square logo 
\usepackage{amssymb}
\usepackage{caption}
\usepackage{upgreek}
\usepackage{subfig}

\usepackage{packages/fancyhdr}% http://ctan.org/pkg/fancyhdr
\pagestyle{fancy}% Change page style to fancy
%\fancyhf{}% Clear header/footer
\fancyhead{}
\fancyhead[RO, LE]{Current monitoring}
%\fancyfoot{}
%\fancyfoot[RO, LE]{\thepage}% \fancyfoot[R]{\thepage}
\renewcommand{\headrulewidth}{0.7pt}% Default \headrulewidth is 0.4pt
%\renewcommand{\footrulewidth}{0.7pt}% Default \footrulewidth is 0pt
%\rfoot{\thepage}

\begin{document}
\baselineskip=16pt
% ---------------------------------------------------------------------------------------------------------------



% ---------------------------------------------------------------------------------------------------------------
\chapter{Current monitoring}
% ---------------------------------------------------------------------------------------------------------------
	- Real-time particle identification
		- Pulse shape (width, area. . .) and its constrants
		- Device constraints
		- Vienna TRIGA reactor
//		- Real time cross section measurement at IRMM?





% ---------------------------------------------------------------------------------------------------------------
\clearpage
\section{Real-time particle identification}
\label{sec:rtpi}
% ---------------------------------------------------------------------------------------------------------------


\subsection{Pulse parameters}
\subsection{Real-time pulse shape analysis algorithm}
\subsection{Device specifications and constraints}
Lab measurements
\subsection{Performance results}
\subsubsection{Thermal neutron measurements}


%\chapter{Particle discrimination}
%
%\section{Discrimination based on TCT pulses}
%
%\subsection{Ionisation profiles in diamond}
%
%
%\subsection{Discrimination parameters}
%FWHM, TCT falling edge (2 pg)
%
%
%\section{Real-time pulse parametrisation}
%(4 pg)
%
%\subsection{Readout device}
%(2 pg)
%
%\subsection{Pulse shape analysis algorithm}
%(8 pg)
%
%\subsection{Device specifications}
%(5 pg)
%
%\subsection{Application: Thermal neutron counter}
%(10 pg)
%
%
%\section{}




% ---------------------------- comment out when compiling the full document -------------------
\end{document}
% ---------------------------------------------------------------------------------------------------------------
