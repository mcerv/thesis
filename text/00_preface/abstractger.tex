\begin{abstractger}

Synthetisch hergestellte Diamanten (Microwave-Assisted Charge Vapour Deposition) k\"onnen benutzt werden, um hochenergetische Teilchen wie $\upalpha$, $\upbeta$ und $\upgamma$ Strahlung, aber auch schnelle und langsame Neutronen, zu detektieren. Dank der hohen Strahlenresistenz und dem exzellente Rauschverhalten sind Diamantdetektoren sehr gute Kandidaten f\"ur Anwendungen in hochenergie Experimenten zum Z\"ahlen von Teilchen und f\"ur die Strahlkontrolle.

Dieses Dokument pr\"asentiert eine Bestrahlungsstudie, die zur Untersuchung der Limitationen von Diamant hinblicklich der Temperaturumgebung und Strahlenh\"arte durchgef\"uhrt wurde. Zun\"achst wurde ein Schadenfaktor von $k_{\mathrm{\lambda}}=(4.4\pm1.2)\times10^{-18}~\upmu$m$^{-1}$~cm$^{2}$ f\"ur 300 MeV Pionen in Einkristalldiamanten gemessen. Danach werden Langzeitstabilit\"atsmessungen f\"ur $\upbeta$ und $\upalpha$ Strahlung auf bestrahlten Mehrkristalldiamanten pr\"asentiert. 
$\upbeta$ Teilchen verbessern das Ladungssammlung von bestrahlten sCVD Diamanten, wohingegen $\upalpha$ Teilchen es verschlechtern. Um dieser Verringerung der Ladunssammlung entgegen zu wirken und die urspr\"ungliche Ladunssammlung wieder herzustellen, werden mehrere "Heilungs"-Prozeduren vorgeschlagen. In dieser Arbeit durchgef\"uhrte TCT Messungen deuten auf die Bildung eines komplexen internen elektrischen Feldes aufgrund der Bildung einer Ortsladung im Diamanten hin.

Diese TCT Messungen vor und nach Bestrahlung der Diamantsensoren zeigen, dass die Ladung-str\"agerfangzeit linear mit der Bestrahlungsfluenz steigt. Weiterhin zeigen $\upalpha$-TCT Messungen im Temperaturbereich zwischen Raumtemperatur und 4 K auf unbestrahlten Diamantsensoren, dass bei 4 K nur noch ca. ein Drittel der Signalladung der Signalladung bei Raumtemperatur gesammelt wird. 
%"Die Ladungssammlungseffizienz der bestrahlten Diamantsensoren ist zwischen 5 und 20 Prozent und zeigt h\"oher bei Temperaturen zwischen 4 K und 75 K als bei Temperaturen zwischen 150 K und Raumtemperatur".

Zwei Anwendungen f\"ur Teilchendetektion mit synthetischen Diamanten werden pr\"asentiert. Die erste Anwendung nutzt die Beobachtung der gemessenen Ladung im Diamanten, und die zweite Anwendung nutzt die Beobachtung des Stromes im Diamanten.

Der ATLAS Diamond Beam Monitor (DBM) ist der erste gepixelte Diamantdetektor, der in einem Hochenergiephysikexperiment eingesetzt wird. Vierundzwanzig Diamant-Pixelsensoren in acht Teleskopen zusammengefasst sind im ATLAS Detector eingebaut. Die Geometrie der Teleskope zeigt auf den Kollisionspunkt, und dank der feinen Segmentierung der Pixelsensoren k\"onnen aus dem Strahlhintergrund und aus dem Kollisionspunkt stammende Teilchenspuren unterschieden werden. Dadurch kann f\"ur jede Kollision die Luminosit\"at im ATLAS Experiment gemessen werden. Dieses Dokument beschreibt den Bau und die Inbetriebnahme des DBM, und dieses Kapitel schliesst mit den ersten Daten von realen Proton-Proton Kollisionen.

Die zweite Anwendung nutzt einzelkristalline Diamantsensoren f\"ur Spektroskopiemessungen. Ein Field Programmable Gate Array (FPGA) wird genutzt um Teilchenidentifiation in Echtzeit zu erm\"oglichen. Das System analysiert die Profile des Stromsignales in Diamantsensoren f\"ur unterschiedliche einfallende Teilchensorten und erm\"oglicht so die Sorte jedes einfallenden Teilchens in Echtzeit zu bestimmen. Es ist in der Lage, bis zu $6\times10^6$~s$^{-1}$ Teilchen zu detektieren und zu bestimmen. Nutzerdefinierte Qualifizierungsvariablen erm\"oglichen eine Feineinstellung zur besseren Teilchensortenbestimmung und k\"onnen histogrammiert werden. Diese Technik wird nun f\"ur Messungen in Umgebungen mit extrem hohem Fluss unterschiedlicher Teilchensorten angewendet, wie z.B. in Neutronenreaktoren. Die Daten in dieser Arbeit wurden an den Reaktoren CROCUS an der EPFL, Schweiz, und TRIGA am Atominstitut in Wien, \"Osterreich aufgenommen.
\end{abstractger}