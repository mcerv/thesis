\begin{abstract}

Diamond produced with the Microwave-Assisted Charge Vapour Deposition technique can be used to detect highly energetic particles ranging from $\upalpha$, $\upbeta$ and $\upgamma$ to fast and slow neutrons. Its high radiation hardness and excellent noise characteristics make it a viable candidate for use in high-energy physics experiments in applications ranging from particle counting to radiation monitoring.

%This document aims to showcase the capabilities and limitations of diamond sensors for high-energy particle detection. First, the general properties of charge carrier behaviour in diamond are outlined. 
This document presents an irradiation study carried out to define the measurement limitations for diamonds in terms of temperature and irradiation damage. First a damage factor $k_{\mathrm{\lambda}}=(4.4\pm1.2)\times10^{-18}~\upmu$m$^{-1}$~cm$^{2}$ is obtained for 300~MeV $\uppi$ irradiated single-crystal CVD diamonds. Then the long-term stability for $\upbeta$ and $\upalpha$ measurements with irradiated diamonds is presented, showing a stable charge collection of the former when the diamond is in a primed state and a significant decrease in charge collection for the latter after a few thousand counts. To counteract the deteriorating $\upalpha$ measurements, several ``healing'' procedures are proposed to recover the charge collection to the initial value. TCT measurements of the long-term $\upalpha$ measurements hint at a development of complex internal electric fields due to accumulated space charge in the bulk of the sensor.

TCT measurements of the diamond sensors before and after irradiation show that charge trapping time increases linearly with irradiation. Furthermore, $\upalpha$-TCT measurements at a range of temperatures between 4~K and room temperature show that the non-irradiated diamond collects only a third of the charge at 4~K as compared to room temperature. The charge collection efficiency of the irradiated diamond increases below 75~K.

Two applications for particle detection with CVD diamond sensors are presented, the first by monitoring collected charge and the second by monitoring the current.

The ATLAS Diamond Beam Monitor is the first pixelated diamond detector used in high-energy physics experiments. It is instrumented with 24 sensors grouped in 8 telescopes that are positioned around the interaction point in the ATLAS experiment. This geometry that points toward the interaction point, together with the high spatial segmentation of the sensors, allows it to distinguish between the beam background and the tracks of particles produced by proton collisions. Its aim is to measure bunch-by-bunch luminosity in ATLAS. This document describes the assembly and testing procedure, as well as integration and commissioning stages. Finally it shows the first data obtained with real proton collisions.

The second application uses sCVD diamond for spectroscopic measurements. It employs the FPGA to carry out real-time particle identification. The system analyses ionisation profiles of the current pulses induced in diamond by incident particles and determines the type of radiation. It is capable of analysing particles at a rate up to $6\times10^6$~s$^{-1}$. By use of user-defined qualifiers it determines the radiation type and fills the respective histogram. The application is tailored for measurements in high-flux environments with mixed types of radiation, such as neutron reactors. The document presents several use cases with data collected in CROCUS at EPFL, Switzerland and TRIGA at Atominstitut, Austria. 





\end{abstract}