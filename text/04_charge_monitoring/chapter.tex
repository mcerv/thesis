% ---------------------------- comment out when compiling the full document -------------------
\documentclass[12pt]{packages/mytustyle}  % default square logo 
\usepackage{amssymb}
\usepackage{caption}
\usepackage{upgreek}
\usepackage{subfig}
\usepackage{textcomp}

\usepackage{packages/fancyhdr}% http://ctan.org/pkg/fancyhdr
\pagestyle{fancy}% Change page style to fancy
%\fancyhf{}% Clear header/footer
%\fancyhead{}
%\fancyhead[RO, LE]{Introduction}	
%\fancyhead[RO, LE]{\rightmark}	
%\fancyfoot{}
%\fancyfoot[RO, LE]{\thepage}% \fancyfoot[R]{\thepage}
\renewcommand{\headrulewidth}{0.7pt}% Default \headrulewidth is 0.4pt
%\renewcommand{\footrulewidth}{0.7pt}% Default \footrulewidth is 0pt
%\rfoot{\thepage}

\begin{document}
\baselineskip=15pt
% ---------------------------------------------------------------------------------------------------------------



% ---------------------------------------------------------------------------------------------------------------
\chapter{Charge monitoring -- the ATLAS Diamond Beam Monitor}
% ---------------------------------------------------------------------------------------------------------------

%-Diamond Beam Monitor
%
%-Spatial segmentation
%
%-FE-I4
%
%-Desc of ATLAS pixel module, functions
%
%-Why do we use diamond in combination with FEI4
%
%-Construction of 24 modules
%
%-Performance results (main part)
%
%-source tests
%
%-desy testbeam (spacial resol, efficiency, TOT)
%
%-Problems, limitations
%
%-Commissioning, installation
%
%-Comparison between diamond and silicon modules.


Particle detectors in high energy physics experiments need to meet very stringent specifications, depending on the functionality and their position in the experiment. In particular, the detectors close to the collision point are subject to high levels of radiation. Then, they need to operate with a high spatial and temporal segmentation to be able to precisely measure trajectories of hundreds of particles in very short time. In addition, they need to be highly efficient. In terms of the structure, their active sensing area has to be thin so as not to cause the particles to scatter or get stopped, which would worsen the measurements. This also means that they have to have a low heat dissipation so that the cooling system dimensions can be minimised. Finally, they need to be able to operate stably for several years without an intervention, because they are buried deep under tonnes of material and electronics. 

The material of choice for the inner detector layers in the HEP experiments is silicon. It can withstand relatively high doses of radiation, it is highly efficient (of the order of $\sim$99.9~\%) and relatively low cost due to using existing industrial processes for its production. Its downside is that, with increasing irradiation levels, it needs to be cooled to increasingly low temperatures to still operate stably. 





%---------------------------------------------------------------------------------------------------------------
%\clearpage
%\section{The ATLAS Diamond Beam Monitor}
%\label{sec:atlasdbm}
%---------------------------------------------------------------------------------------------------------------
The ATLAS Diamond Beam Monitor (DBM) is a novel high energy charged particle detector. Its function is to measure luminosity and beam background in the ATLAS experiment. The monitor's pCVD diamond sensors are instrumented with pixellated FE-I4 front-end chips. The pCVD diamond sensor material was chosen to ensure the durability of the sensors in a radiation-hard environment. The DBM was designed as an upgrade to the existing luminosity monitor called the Beam Conditions Monitor (BCM)~\cite{} consisting of eight diamond pad detectors. It is able to perform precise time-of-flight (ToF) measurements. The DBM complements the BCM's features by implementing tracking capability. Its pixelated front-end electronics significantly increase the spatial resolution of the system. Furthermore, the DBM is able to distinguish particle tracks originating in the collision region from the background hits. This capability is a result of its projective geometry pointing towards the interaction region. This chapter first describes the principles of luminosity measurements. It then explains how the DBM will carry out this task. Finally, some results from tests and from the real collisions are presented. 

When a particle traverses a sensor plane, a hit is recorded in the corresponding pixel. Thus, a precise spatial and timing information of the hit is extracted. With three or more sensors stacked one behind the other, it is also possible to define the particle's trajectory. This is the case with the DBM. Its projective geometry allows the particles to be tracked if they traverse the sensor planes. The DBM relates the luminosity to the number of particle tracks that originate from the collision region of the ATLAS experiment. Particles that hit the DBM from other directions are not taken into account.


%---------------------------------------------------------------------------------------------------------------
%\clearpage
\section{Luminosity measurements}
\label{sec:lummeas}
%---------------------------------------------------------------------------------------------------------------
 \label{sec:lumi}
Luminosity is one of the most important parameters of a particle collider. It is a measurement of the rate of particle collisions that are produced by two particle beams. It can be described as a function of beam parameters, such as: the number of colliding bunch pairs, the revolution frequency, the number of particles in each bunch and the transverse bunch dimensions. The first four parameters are well defined. However, the transverse bunch dimensions have to be determined experimentally during calibration. The ATLAS experiment uses the \emph{van der Meer scan} \cite{} during low-luminosity runs to calibrate the luminosity detectors. This scan is performed by displacing one beam in a given direction and measuring the rate of interactions as a function of the displacement. Transverse charge density of the bunches can be estimated on the basis of the interaction rate. The calibrated luminosity detectors can then operate during high-luminosity runs.

One approach to luminosity monitoring is to count the number of particles produced by the collisions. The luminosity is then proportional to the number of detected particles. A detector has to be capable of distinguishing individual particles that fly from the interaction point through the active sensor area. If the detector has at least three layers, it can reconstruct the particle's track, which gives us more information on the particle's trajectory. This is one reason why detectors with a high time and/or spatial segmentation are more suitable for these applications. The second reason is that, with a high spatial segmentation, the detector will not saturate even at high particle fluences.





%---------------------------------------------------------------------------------------------------------------
%\clearpage
\section{Diamond pixel module}
\label{sec:atlasdbm}
%---------------------------------------------------------------------------------------------------------------
\subsection{pCVD diamond sensors}
\subsection{Silicon sensors}


\subsection{Front-end electronics}
The FE-I4 (front-end version four) is an ASIC pixel chip designed specifically for the ATLAS pixel detector upgrade. It is built as a successor to the current pixel chip FE-I3, surpassing it in size of the active area (4$\times$ larger) as well as the number of channels/pixels (10$\times$ more). 336 such FE-I4 modules are used in the newly installed pixel layer called the Insertable B-Layer (IBL)~\cite{}, as well as the DBM. The FE-I4's integrated circuit contains readout circuitry for 26880 pixels arranged in 80 columns on a 250~$\upmu$m pitch and 336 rows on a 50~$\upmu$m pitch. The size of the active area is therefore 20.0$\times$16.8~mm$^2$. This fine granularity allows for a high-precision particle tracking. The chip operates at 40~MHz with a 25~ns acquisition window, which corresponds to the spacing of the particle bunches in the LHC. It is hence able to correlate hits/tracks to their corresponding bunch. Furthermore, each pixel is capable of measuring the deposited charge of a detected particle by using the Time-over-Threshold (ToT) method. Finally, the FEI4 has been designed to withstand a radiation dose up to 300 MGy. This ensures a longterm stability in the radiation hard forward region of the ATLAS experiment.

\begin{figure}[!t]
\centering
\includegraphics[width=0.45\textwidth]{pics/fei41}
\caption{FE-I4 layout, top-down view. The pink area are pixels grouped into columns, the green area below is the common logic and the red strip at the bottom are the wire bond pads.}
\label{fig:anapix}
\end{figure}


Each pixel is designed as a separate entity. Its electrical chain is shown in figure~\ref{fig:anapix}. The bump-bond pad -- the connection to the outside of the chip -- is the input of the electrical chain, connected to a free-running amplification stage with adjustable shaping using a 4-bit register at the feedback branch. The analog amplifier is designed to collect negative charge, therefore electrons. The output is routed through a discriminator with an adjustable threshold. This value in effect defines the level at which the circuit will detect a hit. In addition, there is a counter of the clock cycles (25~ns sampling) during which the signal is above the discriminator threshold. The value of the counter is proportional to the collected charge. The logic gates at the end of the chain are used to enable/disable the pixel and to issue a so-called HitOr flag -- this signal is set whenever at least one of the pixels was hit and is used as a trigger for the readout. The output of the chain -- HitOut -- is routed into the chip's logic where it is buffered and eventually sent out to the readout system. The module receives all its commands from the system via a 40~MHz LVDS line. The commands are either settings for the pixel registers or triggers that start the data readout. The data are sent via an LVDS line at up to 320~Mbit/s, but by default at 160~Mbit/s, four times faster than the device clock. This enables the device to clear out its buffers before new data are recorded, thus avoiding dead time. The FE-I4 has been successfully tested for trigger rates of up to 300~kHz. 

%The pixels are designed to collect the negative charge.
\begin{figure}[!t]
\centering
\includegraphics[width=0.7\textwidth]{pics/analogPix}
\caption{Schematic of an analog pixel. Courtesy of the FE-I4 collaboration.}
\label{fig:anapix}
\end{figure}

%describe threshold and IF tuning
The DBM uses pCVD diamond with $d_C=500~\upmu$m thickness and silicon with $d_{Si}=200~\upmu$m thickness as a sensor material. The resulting most probable value (MPV) of the deposited charge for a minimum ionising particle (MIP) is calculated with the formula $Q_S=d \cdot E_{e-h}$ and equals 18000~electrons and 17800~electrons, respectively, at a full charge collection efficiency. Unfortunately this is not the case with the pCVD material, whereby we can expect the charge collection efficiency to be of the order of 50~\%  -- around 9000~e. These values further decrease with received irradiation dose. Therefore in order to detect the particles depositing energy on the far left side of the landau spectrum, the threshold has to be set to a significantly lower value. On the other hand, too low threshold means that the electronic noise will trigger a false hit. The typical noise amplitudes being in the range of 120--200~e, a safe threshold range would be between $Th=$ 1000--3000~e.

The analog amplifier is implemented in two stages to get a fast rise time at a low noise and a low power consumption. The output signal of the analog amplifier has a triangular shape with a fast rise time and a long decay.  The shape can be adjusted by tuning the amplifier feedback loop. Its length is proportional to the collected charge, but it needs to be calibrated first. This is done using the injection capacitors $C_{inj1}$ and $C_{inj2}$ seen in figure~\ref{fig:anapix} with well defined capacitances. A charge $Q_{cal}=V_{cal}\cdot(C_{inj1}+C_{inj2})$ is injected into the analog chain, the length of the output pulse is measured and the feedback value is changed to either lengthen or shorten the pulse in order to get to the required duration $t_{cal}$. The typical values are $Q_{cal}=5000-16000~$e at the time $t_{cal}=5-10~$ToT, depending on the sensor, radioactive source and application. Therefore the initial threshold $Th$ at 1~ToT and the calibrated value $Q_{cal}$ at $t_{cal}$~ToT give us a linear scale of collected charge with respect to time over threshold.
However, in practice this relation is nonlinear for lower thresholds, but since the goal of the measurements is to track the particles rather than to measure their deposited energy precisely, this is sufficient. 
 %The target values depend on the type of the sensor and the intended use of the device.




















% It is capable of carrying out position-resolved measurements by performing high-precision particle tracking.
%Currently, the ATLAS pixel detector consists of three layers of silicon pixel sensors utilising the FE-I3 front-end ASIC pixel readout chips. In order to increase the impact parameter resolution of the detector, a fourth layer of sensors will be installed around the beam pipe. 


%---------------------------------------------------------------------------------------------------------------
\subsection{Module assembly and quality control}
%\label{sec:modass}
%---------------------------------------------------------------------------------------------------------------
Parts for the detector arrived separately and were assembled into modules at CERN's DSF lab after being checked for production faults. The assembled modules underwent a series of quality control (QC) and burn-in tests to determine their quality, efficiency and long-term stability.
\subsubsection{Assembly}
A single-chip module consists of a pixel module, a flexible PCB and the supporting mechanics (a ceramic plate and an aluminium plate). The chip arrives already bump-bonded to the sensor, be it diamond or silicon. First it is glued to the ceramic plate on one side and to the PCB on the other using either Araldite 2011 or Staystik 672/472. The choice of glue was a topic of a lengthy discussion. Staystik is re-workable and has a very high thermal conductivity. The latter is important because the FE-I4 chips tend to heat up significantly and need a good heat sink. The problem is that it has a curing temperature of 160/170~\textdegree C. This temperature could cause some unwanted tension build-up between the FE-I4 and the diamond sensor due to different coefficients of thermal expansion, disconnecting regions of pixels. To avoid this, an alternative glue was tried. Araldite~2011 can be cured at lower temperatures -- down to RT -- but it has a lower heat conductivity. In the end Araldite was chosen as the safer option. However, due to the longer curing, the whole assembly process using Araldite is extended to two working days. After curing, the module is wire-bonded and attached to the aluminium plate using screws made up of a radiation-resistant PEEK plastic. They have to be tightened with a great care, because their screw head is only 0.2--0.6~mm away from the sensor edge -- the sensor displacement tolerance during gluing is of the order of 0.5~mm. Finally, the module is put in an aluminium carrier which protects the module from mechanical damage or electrostatic discharges.

\subsubsection{Testing}
The modules were tested in the lab using an RCE readout system and a moving stage with two degrees of freedom. They were placed onto the stage and connected to the readout system and the power supplies. After ensuring the low- and high voltage connectivity they were checked for the signal connectivity. If everything was operational, a series of automated tests was run. Each of these tests calibrates a certain value within a pixel, whether it is the signal threshold or the value for integrated charge. These are tuned in a way that the response to a predefined calibration signal is uniform for all pixels across the sensor. This procedure is referred to as \emph{tuning}. 

When the modules were tuned, they were tested using a $^{90}$Sr radioactive source. Two things were tested: 1) operation of all pixels and 2) sensor efficiency. The first test was carried out by moving the module slowly under the source while taking data so that the whole surface was scanned uniformly. The resulting occupancy map revealed any pixels that were not electrically connected to the sensor via bump bonds. This was an important step in the DBM QC procedure, because it turned out that a significant portion of the flip-chipped diamond sensors exhibited very poor connectivity. The disconnected regions on the faulty modules ranged anywhere from 0.5--80~\% of the overall active surface. In two cases the sensor was even completely detached from the chip. The pixel connectivity turned out to be the most important qualification factor in the QC procedure. Unfortunately the only way to check it was to fully assemble a module and test it using a radioactive source. If the module turned out to be of poor quality, it was disassembled and sent for rework. The turnover time of this operation was of the order of one month, which affected the detector installation schedule significantly. Only the sensors that passed the pixel connectivity test underwent the second test stage in which the sensor's efficiency was estimated. In principle, a scintillator placed underneath the module was used as a trigger; a particle that crossed the DBM module and hit the scintillator, triggered the module readout. In the end, the number of triggers was compared to the number of hits/clusters recorded by the module. The resulting ratio was an estimate of the sensor's detection efficiency. The real sensor efficiency can only be measured in a particle beam and using a beam telescope as a reference detector. Nonetheless, the so-called \emph{pseudo-efficiency} gave a rough estimate of the sensors' quality. The results are shown in section~\ref{sec:perfresults}.  All in all, 79 modules went through the QC procedure -- 45 diamond modules and 34 silicon modules, 12 of the latter only for testing purposes. 18 diamond modules and 6 silicon modules were in the end chosen to be made up into DBM telescopes and installed into ATLAS.

A very important issue was the so called erratic current. This term describes the leakage current in a pCVD diamond that becomes unstable. It can develop gradually or can be triggered with a $\upbeta$ source. Spikes appear in the otherwise stable leakage current. They can be up to three orders of magnitude higher than the base current. Sometimes the current also suddenly increases for a few orders of magnitude and stays at that level (e.g. from the initial 1~nA to 3~$\upmu$A). The amplitude differs in magnitude from sensor to sensor. This effect is still not fully explained, but the hypothesis is that the charges find a conductive channel along the grain boundaries, causing discharges. These discharges are picked up by the pixel amplifiers in the FE-I4. A single discharge can trigger a group of up to $\sim$500 pixels, resulting in a \emph{blob} on the detector occupancy map. Sometimes the conductive channel stays in a conductive state, making one or more pixels always to fire. These pixels are useless and have to be masked out during measurements. 

\subsection{Installation and commissioning}
The DBM modules that passed the QC tests were assembled into telescopes -- sets of three modules one behind the other with a spacing of 50~mm. Of the 18 diamond and 6 silicon modules, 6 diamond and 2 silicon modules were built. A special care was taken when choosing the sets of three diamonds. The modules with a similar pseudo-efficiency, leakage current, maximum stable high voltage and shape of disconnected regions were grouped together. After assembly into telescopes, the modules were tested for their connectivity. Then the high voltage was applied and the leakage current was observed. This was an important point to check because all three modules shared the same high voltage channel. Any instabilities on one of the modules would cause problems on the other two. This would for instance happen if one of the modules had a much lower breakdown voltage.

Due to time constraints, the telescopes were not built at the same time but instead in a pipeline. As soon as two telescopes were ready, they were transported to Point~1 -- the ATLAS site. There they were prepared for installation onto the pixel detector structure that had been extracted from ATLAS due to pixel detector commissioning. The commissioning was nearing the end, so the technicians were preparing the detector for re-insertion. The cylindrical structure was being closed off by four new service quarter-panels (nSQPs). This meant that with every day the access to the place of installation of the DBM was more difficult to reach. The first two telescopes were put into place when only one nSQP was in place. This allowed the process to be conducted from both sides. This proved to be helpful, because the process was lengthy and had to be done with great precision. It involved tightening of several screws on both sides of the telescopes. At the same time the surrounding electronics and cables had to be left untouched. The lessons learnt with the first part of the installation were helpful when installing the other telescopes. The last two were fitted onto the structure when three nSQPs were already in place, leaving only a narrow opening for access. The whole procedure was carried out blind. After every installation, the telescopes were tested again. First, the low voltage connectivity was checked and a set of tests was run on the FE-I4 front-end chips. An eye diagram was made to estimate the quality of the signal transmission. Then a $^{90}$Sr source was used to perform a source test on three modules at the same time. Leakage current was observed during the source test. The final test included running four telescopes (all on one side) at a time. All the tests were successful and the DBM was signed off.


%---------------------------------------------------------------------------------------------------------------
\section{Performance results}
\label{sec:perfresults}
%---------------------------------------------------------------------------------------------------------------
This section gives an overview of the performance results of the DBM modules achieved during the QC and the test beam campaign. The source tests were performed to check for disconnected regions in the sensors and to measure the diamond's pseudo-efficiency. Only the modules with minimal disconnected regions and maximum pseudo-efficiency were chosen for installation. 
\subsection{Source tests}
All modules went through the same procedure when tested using a $^{90}$Sr source -- to check for disconnected regions and to measure the pseudo-efficiency. The majority of the silicon modules yielded the pseudo-efficiency of 94.3$\pm$0.2~\%. Silicon sensors being 99.99~\% efficient, this value vas underestimated by about 5~\%. The measured pseudo-efficiency of the diamond modules was anywhere between 5--80~\%, depending on the diamond quality, the set threshold and the applied bias voltage. These three settings were varied to check the behaviour of the modules under various conditions. 
\subsection{Test beam results}
The first two assembled prototype DBM modules, MDBM-01 and MDBM-03, were tested at DESY, Hamburg, in a test beam facility. The aim of the measurements was to measure their efficiency, the spatial distribution of the efficiency and the effect of the beam on the disconnected regions. A silicon module MSBM-02 was measured to crosscheck the measurements. Since the silicon module is almost 100~\% efficient, it was used as an "anchor" -- the diamond's efficiency was measured relative to the silicons' efficiency. Two beam telescopes were used as reference systems: Kartel~\cite{}, built by JSI institute from Ljubljana, and EUDET Aconite~\cite{}. Both are instrumented with six Mimosa26 pixel planes and capable of tracking particles with a 2~$\upmu$m tracking resolution.

The test beam prototypes did not meet the acceptance criteria for production DBM modules in the following areas: first, the CCDs were slightly below 200~$\upmu$m, which would be the DBM minimum. Secondly, the applied bias voltages ranged from 1--2~V/$\upmu$m. In addition, the threshold cut could only be set to 1500~electrons, which is higher than the DBM minimum (1000~e). Nonetheless, the resulting module efficiencies were still in the range between 75--85~\%. The 

Judith~\cite{} is a software framework for analysing test beam data. It is capable of synchronising data streams from several detector systems only connected via a trigger system, reconstructing tracks and calculating efficiency for the DUTs. It was also used to reconstruct and analyse the acquired Kartel test beam data together with the silicon and diamond module as DUTs.A sample of the analysed data is shown in figure~\ref{}. 

Spatial resolution, efficiency, ToT


%---------------------------------------------------------------------------------------------------------------
\section{Operation}
\label{sec:operation}
%---------------------------------------------------------------------------------------------------------------

\subsection{Positioning}
The DBM is placed in the forward region of the ATLAS detector, very close to the beam pipe. The mechanical structure that holds the sensor planes is, due to its shape, referred to as a DBM telescope. A telescope is a system that consists of several pixel sensors placed in series one behind the other. Each DBM telescope houses three diamond pixel modules. Eight DBM telescopes reside approximately 1~m away from the collision region, four on each side. They are tilted with respect to the beam pipe for 10\textdegree. This is due to a specific phenomenon connected to erratic (dark) currents in diamond. Studies have shown~\cite{} that the erratic leakage currents that gradually develop in diamond can be suppressed under certain conditions. For instance, if a strong magnetic field is applied perpendicular to the electric field lines in the diamond bulk, the leakage current stabilises~\cite{}. The DBM was designed to exploit this phenomenon. The magnetic field lines in the ATLAS experiment are parallel to the beam. Hence, an angular displacement of the sensor with respect to the beam allows for the leakage current suppression. However, the DBM telescopes still need to be directed towards the interaction region. Taking these considerations into account, a 10\textdegree~angle with respect to the beam pipe was chosen. The influence of the magnetic field on the particle tracks at this angle is very low as the field lines are almost parallel to the tracks. The tracks are therefore straight, which reduces the track reconstruction complexity.


\subsection{Data taking during collisions}
The DBM has been commissioned in ATLAS and is now taking data. Several issues still need to be resolved regarding the readout systems. Unfortunately, due to issues with the low voltage power supply regulators, six out of 24 modules died during operation: four silicon and two diamond modules. The system configured the modules into an unsteady state whereby they drew twice as much current as the allowed maximum. This current fused the wire bonds within minutes. This has left only five diamond telescopes still fully operational. The preliminary data obtained using the remaining telescopes show that the background rejection could indeed work. U






%---------------------------------------------------------------------------------------------------------------
\section{Limitations}
\label{sec:limitations}
%---------------------------------------------------------------------------------------------------------------

comparison between diamond and silicon modules

%---------------------------------------------------------------------------------------------------------------
\section{Conclusion}
\label{sec:limitations}
%---------------------------------------------------------------------------------------------------------------
The Diamond Beam Monitor has been designed as an upgrade to the existing luminosity detectors in the ATLAS experiment. It is the first diamond pixel tracking detector installed in a high-energy physics experiment. The pixelated front-end electronic chips ensure precise spatial detection of the charged high-energy particles. The projective geometry allows for particle tracking and background rejection. The detector is placed in a high-radiation forward region of the experiment. Therefore, radiation hardness of the chosen pCVD diamond sensors is an important requirement. The tests carried out in the test beam and in the laboratory confirmed that the DBM modules were ready to be installed in the experiment. The DBM is now running in ATLAS during collisions. Further improvements have to be made on the readout firmware before it is included in the main readout stream. 





%---------------------------- comment out when compiling the full document -------------------
\end{document}
%---------------------------------------------------------------------------------------------------------------
